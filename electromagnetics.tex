\documentclass{article}
\usepackage{amsmath}
\usepackage{amssymb}
\usepackage{amsfonts}
\usepackage{pgf}

%=============================================================================
%                              Useful Commands
%=============================================================================

\newcommand{\refEq}[1]{(\ref{eq:#1})}
\newcommand{\refChap}[1]{Chapter~\ref{chap:#1}}
\newcommand{\refFig}[1]{Figure~\ref{fig:#1}}
\newcommand{\refSec}[1]{Section~\ref{sec:#1}}
\newcommand{\refTab}[1]{Table~\ref{tab:#1}}
\newcommand{\refApp}[1]{Appendix~\ref{app:#1}}
\newcommand{\newEq}[2]{\begin{equation} \label{eq:#1} #2 \end{equation}}
\newcommand{\cross}{\!\times\!}
\newcommand{\inner}{\!\cdot\!}
\newcommand{\Div}{\nabla\!\cdot\!}
\newcommand{\Divx}{\nabla_{\vec{x}}\!\cdot\!}
\newcommand{\Curl}{\nabla\!\times\!}
\newcommand{\Curlx}{\nabla_{\vec{x}}\!\times\!}
\newcommand{\Grad}{\nabla\!}
\newcommand{\Gradu}{\nabla_{\vec{u}}\!}
\newcommand{\Gradx}{\nabla_{\vec{x}}\!}
\newcommand{\st}{\Bigl\lvert\;}
\newcommand{\St}{\biggl\lvert\;}
%\newcommand{\Div}{\nabla\cdot}
%\newcommand{\Divx}{\nabla_{\vec{x}}\cdot}
%\newcommand{\Curl}{\nabla\times}
%\newcommand{\Curlx}{\nabla_{\vec{x}}\times}
%\newcommand{\Grad}{\nabla}
%\newcommand{\Gradu}{\nabla_{\vec{u}}}
%\newcommand{\Gradx}{\nabla_{\vec{x}}}

\providecommand{\abs}[1]{{\left\lvert#1\right\rvert}}
\providecommand{\norm}[1]{{\left\lVert#1\right\rVert}}

\def\Hdiv{$H(Div)$ }
\def\Hcurl{$H(Curl)$ }

%\pgfdeclareimage[width=5in]{pen_fun}{images/pen_fun_and_deriv}

%=============================================================================

\title{MFEM Electromagnetics Mini Applications}
\author{The MFEM Team}

\begin{document}

\maketitle

\section{Electromagnetics}

The equations describing electromagnetic phenomena are know
collectively as the ``Maxwell Equations''.  They are usually given as:
\begin{eqnarray}
\Curl\vec{H} - \frac{\partial\vec{D}}{\partial t} &=& \vec{J} \label{eq:ampere} \\
\Curl\vec{E} + \frac{\partial\vec{B}}{\partial t} &=& 0 \label{eq:faraday} \\
\Div\vec{D} &=& \rho \label{eq:gauss} \\
\Div\vec{B} &=& 0 \label{eq:divb}
\end{eqnarray}
Where equation~\refEq{ampere} can be referred to as {\em Amp\`ere's Law},
equation~\refEq{faraday} is called {\em Faraday's Law},
equation~\refEq{gauss} is {\em Gauss's Law}, and equation~\refEq{divb}
doesn't generally have a name but is related to the nonexistence of
magnetic monopoles.  The various fields in these equations are:
\begin{center}
\begin{tabular}{|l|l|l|}
\hline
Symbol & Name & SI Units \\
\hline
$\vec{H}$ & Magnetic Field & Ampere/meter \\
$\vec{B}$ & Magnetic Flux Density & Tesla \\
$\vec{E}$ & Electric Field & Volts/meter \\
$\vec{D}$ & Electric Displacement & Coulomb/meter$^2$ \\
$\vec{J}$ & Current Density & Ampere/meter$^2$ \\
$\rho$ & Charge Density & Coulomb/meter$^3$ \\
\hline
\end{tabular}
\end{center}
In the literature these names do vary, particularly those for
$\vec{H}$ and $\vec{B}$, but in this document we will try to adhere to
the convention laid out above.

Generally we also need constitutive relations between $\vec{E}$ and
$\vec{D}$ and/or between $\vec{H}$ and $\vec{B}$.  These relations
start with the definitions:
\begin{eqnarray}
\vec{D} &=& \epsilon_0\vec{E} + \vec{P}\\
\vec{B} &=& \mu_0\left(\vec{H} + \vec{M}\right)
\end{eqnarray}
Where $\vec{P}$ is the {\em polarization density}, and $\vec{M}$ is
the {\em magnetization}.  Also, $\epsilon_0$ is the {\em permittivity
  of free space} and $\mu_0$ is the {\em permeability of free space}
which are both constants of nature.  In many common materials the
polarization density can be approximated as a scalar multiple of the
electric field i.e. $\vec{P}=\epsilon_0\chi\vec{E}$, where $\chi$ is
called the {\em electric suscepctibility}.  In such cases we usually
use the relation $\vec{D}=\epsilon\vec{E}$ with
$\epsilon=\epsilon_0(1+\chi)$ and call $\epsilon$ the {\em
  permittivity} of the material.

The nature of magnetization is more complicated but we will take a
very simplified view which is valid in many situations.  Specifically,
we will assume that either $\vec{M}$ is proportional to $\vec{H}$
yielding the relation $\vec{B}=\mu\vec{H}$ where $\mu=\mu_0(1+\chi_M)$
and $\chi_M$ is the {\em magnetic susceptibility} or that $\vec{M}$ is
independent of the applied field.  The former case pertains to both
diamagnetic and paramagnetic materials and the later to ferromagnetic
materials.

Finally we should note that equations~\refEq{ampere} and \refEq{gauss}
can be combined to yield the equation of charge continuity
\[\frac{\partial\rho}{\partial t}+\Div\vec{J} = 0\]
which can be important in plasma physics and magnetohydrodynamics (MHD).

\subsection{Static Fields}
\subsubsection{Electrostatics}

Electrostatic problems come in a variety of subtypes but they all
derive from Gauss's Law and Faraday's Law (equations~\refEq{gauss} and
\refEq{faraday}).  When we assume no time variation, Faraday's Law
becomes simply $\Curl\vec{E}=0$. This suggests that the electric field
can be expressed as the gradient of a scalar field which is
traditionally taken to be $-\varphi$, i.e.
\begin{equation}
\vec{E} = -\Grad\varphi \label{eq:gradphi}
\end{equation}
where $\varphi$ is called the {\em electric potential} and has units
of Volts in the SI system.  Inserting this definition into
equation~\refEq{gauss} gives:
\begin{equation}
-\Div\epsilon\Grad\varphi = \rho - \Div\vec{P}\label{eq:poisson}
\end{equation}
which is {\em Poisson's equation} for the electric potential.  Where
we have assumed a linear constitutive relation between $\vec{D}$ and
$\vec{E}$ of the form $\vec{D}=\epsilon\vec{E}+\vec{P}$.  This allows
a polarization which is proportional to $\vec{E}$ as well a
polarization independent of $\vec{E}$. If this relation happens to be
nonlinear then Poisson's equation would need to be replaced with a
more complicated nonlinear expression.

The solutions to equation~\refEq{poisson} are non unique because they
can be shifted by any additive constant.  This means that we must
apply a Dirichlet boundary condition at at least one point in the
problem domain in order to obtain a solution.  Typically this point
will be on the boundry but it need not be so.  Such a Dirichlet value
is equivalent to fixing the voltage (aka potential) at one or more
locations.  Additionally, this equation admits a normal derivative
boundary condition.  This means setting $\hat{n}\cdot\vec{D}$ to a
prescribed value on some portion of the boundary.  This is equivalent
to defining a surface charge density on that portion of the boundary.

\subsubsection{{\tt volta} Mini Application}

The electrostatics mini application, named {\tt volta} after the
inventor of the voltaic pile, is intended to demonstrate how to solve
standard electrostatics problems in MFEM.  Its source terms and
boundary conditions are simple but they should indicate how more
specialized sources or boundary conditions could be implemented.  Note
that this application assumes the mesh coordinates are given in
meters.

\begin{description}
\item[Mini Application Features:]
\item[Permittivity:] The permittivity is assumed to be that of free
  space except for an optional sphere of dielectric material which can
  be defined by the user.  The command line option {\tt -ds} can be
  used to set the parameters for this dielectric sphere.  For example,
  to produce a sphere at the origin with a radius of 0.5 and a
  relative permittivity of 3 the user would specify:
  \begin{center}{\tt -ds '0 0 0 0.5 3'}\end{center}

\item[Charge Density:] The charge density is assumed to be zero except
  for an optional sphere of uniform charge density which can be
  defined by the user.  The command line option for this is {\tt -cs}
  which follows the same pattern as the dielectric sphere.  Note that
  the last entry is the total charge of the sphere and not its charge
  density.

\item[Polarization:] A polarization vector function can be imposed as
  a source of the electric field.  The command line option {\tt -vp}
  creates a polarization due to a simple voltaic pile i.e. a cylinder
  which is electrically polarized along its axis.  The user should
  specify the two end points of the cylinder axis, its radius and the
  magnitude of the polarization vector.

\item[Dirichlet BC:] Dirichlet Boundary Conditions can either specify
  piecewise constant voltages on a collection of surfaces or they can
  specify a gradient field which approximates a uniform applied
  electric field.  In either case the user specifies the surfaces
  where the Dirichlet boundary condition should be applied using the
  {\tt -dbcs} option followed by a list of boundary attributes.
  For example to select surfaces 2, 3, and 4 the user would use the
  following:
  \begin{center}{\tt -dbcs '2 3 4'}\end{center}
  To apply a gradient field on these surfaces the
  user would also use the {\tt -dbcg} option.  This defaults to the
  uniform field $\vec{E}=(0,0,1)$ in 3D or $\vec{E}=(0,1)$ in 2D.  An
  arbitrary vector can be with {\tt -uebc} followed by the desired
  vector e.g. to apply $\vec{E}=(1,2,3)$ the user would supply:
  \begin{center}{\tt -uebc '1 2 3'}\end{center}
  To specify piecewise constant potential values the user would list
  the desired values after {\tt -dbcv} as follows:
  \begin{center}{\tt -dbcv '0.0 1.0 -1.0'}\end{center}

\item[Neumann BC:] Neumann Boundary Conditions set the normal
  component of the electric displacement on portions of the boundary.
  This normal component is equivalent to the surface charge density on
  the surface.  This is rarely used because surface charge densities
  are rarely known unless they are know to be zero.  However, if the
  surface charge density is zero then the Neumann BCs are not needed
  because this is the natural boundary condition.  Only piecewise
  constant Neumann boundary conditions are supported.  They can be set
  analygously to piecewise Dirichlet boundary conditions but using
  options {\tt -nbcs} and {\tt -nbcv}.

\end{description}


\subsubsection{Magnetostatics}

Magnetostatic problems arise when we assume no time variation in
Amp\`ere's Law (equation~\refEq{ampere}) which leads to:
\[\Curl\vec{H}=\vec{J}\]
We will again assume a somewhat more general constitutive
relation between $\vec{H}$ and $\vec{B}$ than is normally seen:
\[\vec{B}=\mu\vec{H}+\mu_0\vec{M} = \mu_0\left(1+\chi_M\right)\vec{H}+\mu_0\vec{M}\]
Where the magnetization is split into two portions; one which is
proportional to $\vec{H}$ and given by $\chi_M\vec{H}$, and another
which is independent of $\vec{H}$ and is given by $\vec{M}$.  This
allows for paramagnetic and/or diamagnetic materials defined through
$\mu$ as well as ferromagnetic materials represented by $\vec{M}$.
This choice yields:
\[\Curl\mu^{-1}\vec{B}=\vec{J}+\Curl\mu^{-1}\mu_0\vec{M}\]
Which, when combined with equation~\refEq{divb} becomes:
\begin{equation}
\Curl\mu^{-1}\Curl\vec{A}=\vec{J}+\Curl\mu^{-1}\mu_0\vec{M}
\end{equation}
If $\vec{J}$ happens to be zero we have another option because we can
assume that $\vec{H} = -\Grad\varphi_M$ for some scalar potential
$\varphi_M$.  When combined with equation~\refEq{divb} becomes:
\begin{equation}
\Div\mu\Grad\varphi_M = \Div\mu_0\vec{M}
\end{equation}
Currently only the vector potential equation is used so we will focus
on that for the remainder of this documentation.

The vector potential is again non unique so we must apply additional
constraints in order to arrive at a solution for $\vec{A}$.  When
working analytically it is common to constrain the solution by
restricting the divergence of $\vec{A}$ but numerically this leads to
other complications.  For our problems of interest it will be
necessary to require Dirichlet boundary conditions on the entire outer
surface in order to sufficiently constrain the solution.

Dirichlet boundary conditions for the vector potential on a surface
provide a means to specify the component of $\vec{B}$ normal to that
surface.  For example, setting the tangential components of $\vec{A}$
to be zero on a particular surface results in a magnetic flux density
which must be tangent to that surface.

\subsubsection{{\tt tesla} Mini Application}

The magnetostatics mini application, named {\tt tesla} after the unit
of magnetic field strength (and of course the man Nikola Tesla), is
intended to demonstrate how to solve standard magnetostatics problems
in MFEM.  Its source terms and boundary conditions are simple but they
should indicate how more specialized sources of boundary conditions
could be implemented.  Note that this application assumes the mesh
coordinates are given in meters.

\begin{description}
\item[Mini Application Features:]
\item[Permeability:] The permeability is assumed to be that of free
  space except for an option spherical shell of diamagnetic or
  paramagnetic material which can be defined by the user.  The command
  line option {\tt -ms} can be used to set the parameters for this
  shell.  For example, to produce a shell at the origin with inner and
  outer radii of 0.4 and 0.5 respectively and a relative permeability
  of 3 the user would specify:
  \begin{center}{\tt -ms '0 0 0 0.4 0.5 3'}\end{center}

\item[Current Density:] The current density is assumed to be zero
  except for an optional ring of constant current which can be defined
  by the user.  The command line option for this is {\tt -cr} which
  requires two points giving the end points of the ring's axis, inner
  and outer radii, and a constant total current.  For example to
  specify a ring centered at the origin and laying in the xy plane
  with a thickness of 0.2 and radii 0.4 and 0.5, and a current of 2
  amps the user would give:
  \begin{center}{\tt -cr 0 0 -0.1 0 0 0.1 0.4 0.5 2}\end{center}

\item[Magnetization:] A permanent magnetization can be applied in the
  form of a cylindrical magnet with poles at its circular ends.  The
  command line option is {\tt -bm} which indicates a 'bar magnet'.
  The option requires the two end points of the cylinder's axis, its
  radius, and the magnitude of the magnetization.

\item[Surface Current Density:] A surface current can be imposed
  indirectly by specifying separate surface patches with different
  voltages as well as a collection of surface patches connecting the
  voltages through which the current will flow.  The voltage surfaces
  and their voltages can be specified using {\tt -vbcs} followed by
  the indices of the surfaces and {\tt -vbcv} followed by their
  voltages.  The path for the surface current ($\vec{K}$) is specified
  by using {\tt -kbcs} followed by a set of surface indices.  For
  example applying voltages 1 and -1 to surfaces 2 and 3 with a
  current path along surfaces 4 and 6 would be specified as:
  \begin{center}{\tt -vbcs '2 3' -vbcv '1 -1' -kbcs '4 6'}\end{center}
  Any surfaces not listed as voltage or current surfaces will be
  assigned as homogeneous Dirichlet boundaries.  Note that when this
  option is selected an auxilliary electrostatic problem will be
  solved on the surface of the geometry to compute the surface
  current.

\item[Dirichlet BC:] Dirichlet Boundary Conditions are required if a
  surface current density is not defined.  For this reason the user
  need not specify boundary surfaces by number since the boundary
  condition must be applied on all of them.  The default boundary
  condition is a homogeneous Dirichlet boundary condition on all outer
  surfaces.  This means that the normal component of $\vec{B}$ will be
  zero at the outer boudnary.  An alternative is to specify a desire
  uniform magnetic flux density on the entire outer surface.  This is
  accomplished with the {\tt -ubbc} command line option followed by
  the desired $\vec{B}$ vector.
\end{description}


\subsection{Dynamic Fields}
\subsubsection{Transient MAgnetics and Joule Heating}
The transient magnetics mini application, named `Joule` after the SI unit of energy (and the
scientist James Prescott Joule, who was also a brewer), is intended to demonstrate how to solve
transient implicit diffusion problems. The equations of low-frequency electromagnetics are coupled
with the equations of heat transfer. The coupling is one way, electromagnetics generates Joule
heating, but the heating does not affect the electromagnetics.  The thermal problem
is solved using an $H(\\mathrm{div})$ method, i.e. temperature is discontinuous and the
thermal flux $\F$ is in $H(\\mathrm{div})$.
There are three linear solves per time step:

1. Poisson's equation for the scalar electric potential is solved using the AMG
 preconditioner,
2. the electric diffusion equation is solved using the AMS preconditioner, and
3. the thermal diffusion equation is solved using the ADS preconditioner.

Two example meshes are provided, one is a straight circular metal rod in vacuum, the other is a helical
coil in vacuum. The idea is that a voltage is applied to the ends of the rod/coil, the electric field diffuses
into the metal, the metal is heated by Joule heating, the heat diffuses out.

The equations are:

\begin{eqnarray}
    \Div\sigma\grad\Phi &= 0 \\
    \sigma \E &= \Curl\mu^{-1} \B - \sigma \Grad \Phi \\
    \frac{d \B}{d t} &= - \Curl \E \\
    \F &= -k \Grad T \\
    c \frac{d T}{d t} &= - \Div \F + \sigma \E \cdot \E
\end{eqnarray}

The equations are integrated in time using implicit time integration, either midpoint or
higher order SDIRK.

Since there are three solves,  three sets of boundary conditions must be specified. The
essential BC's are the scalar potential, the electric field, and the thermal flux. These are not
set via command line arguments, you have to edit the code to change these. To change these,
search the code for `ess_bdr`

There are conducting and non-conducting material regions, and the mesh must have integer attributes
to specify these regions. To change these, search the code for `std::map<int, double>` this maps the
integer attribute to the floating-point material value.

Note that this application assumes the mesh coordinates are given in meters.

\begin{description}
\item[Mini Application Features:]

\item[Boundary Conditions:] Since there are three solves,  three sets of boundary conditions must be specified. The
essential BC's are the voltage for the scalar potential, the tangential electric field, and the normal thermal flux.
These are not
set via command line arguments, you have to edit the code to change these. To change these,
search the code for `ess_bdr`. Note that the essential BC's can be time varying.

\item[Material Properties:] There are conducting and non-conducting material regions, and the mesh must have integer attributes
to specify these regions. To change these, search the code for `std::map<int, double>` this maps the
integer attribute to the floating-point material value.
\end[description


\end{document}
